\documentclass{article}\usepackage{makeidx}\makeindexegin{document}\printindex
\section{main}
% Generated by sldoc (1.14, (C) 1996, 1997) Tue Jan 15 14:51:25 2019


\paragraph{\index{addtohist}addtohist}
\begin{verbatim}
int addtohist(char *item)
\end{verbatim}
adds {\tt item} to the history list
*\%


from: {\tt hist.c}

from file: {\tt doc}


\paragraph{\index{addtotmplist}addtotmplist}
\begin{verbatim}
int addtotmplist(char *item)
\end{verbatim}
adds {\tt item} to the history list
*\%


from: {\tt tmplist.c}

from file: {\tt doc}


\paragraph{\index{cktype}cktype}
\begin{verbatim}
int cktype (char *fmt)
\end{verbatim}
Checks type that should be pushed on stack
Returns one of the SC\_* defines


from: {\tt scanf.c}

from file: {\tt doc}


\paragraph{\index{del\_tmplist}del\_tmplist}
\begin{verbatim}
void del_tmplist(int files)
\end{verbatim}
Deletes the files in the tmp list if {\tt files} $>$ 0. Otherwise only
the list itself is freed 


from: {\tt tmplist.c}

from file: {\tt doc}


\paragraph{\index{disclaim}disclaim}
\begin{verbatim}
void disclaim(char *progname)
\end{verbatim}
shows a short version of the GNU public licence on stderr


from: {\tt vamps.c}

from file: {\tt doc}


\paragraph{\index{dorun}dorun}
\begin{verbatim}
Void dorun()
\end{verbatim}
Does the main body of the model run. From the first to the last
timestep.


from: {\tt vamps.c}

from file: {\tt doc}


\paragraph{\index{dotail}dotail}
\begin{verbatim}
Void dotail()
\end{verbatim}
Does 'tail' part of a model run. Cleans op files and memory. Also saves
some stuff to disk.


from: {\tt vamps.c}

from file: {\tt doc}


\paragraph{\index{init\_s}init\_s}
@void init\_s (char *vamps\_sl)


Initializes the S-Lang system and loads the startup file
{\tt vamps\_sl}.


from: {\tt init\_s.c}

from file: {\tt doc}


\paragraph{\index{lai\_to\_s}lai\_to\_s}
\begin{verbatim}
double lai_to_s (double lai)
\end{verbatim}
Converts lai to canopy storage by a user defined function in s\_lang
called Slai\_to\_s (lai).


from: {\tt init\_s.c}

from file: {\tt doc}


\paragraph{\index{loaddefaults}loaddefaults}
\begin{verbatim}
void loaddefaults ()
\end{verbatim}
loads defaults from the defaults-file. After that the defaults
in the vamps section from the input file are loaded


from: {\tt vamps.c}

from file: {\tt doc}


\paragraph{\index{main}main}
\begin{verbatim}
int main (int argc, char *argv[])
\end{verbatim}
The main vamps program


from: {\tt vamps.c}

from file: {\tt doc}


\paragraph{\index{onsig}onsig}
\begin{verbatim}
void onsig (int sig)
\end{verbatim}
New signal handler. Setup is run again after handling the signal as
Linux resets them to default behaviour. I suppose this is no
problem on other systems ;-).
Under MSDOS (djgpp) the stuff does not seem to work (what's new). 


from: {\tt sigs.c}

from file: {\tt doc}


\paragraph{\index{prelim}prelim}
\begin{verbatim}
void prelim(void)
\end{verbatim}
from: {\tt vamps.c}

from file: {\tt doc}


\paragraph{\index{processkey}processkey}
\begin{verbatim}
void processkey(char ch)
\end{verbatim}
from: {\tt vamps.c}

from file: {\tt doc}


\paragraph{\index{pushit}pushit}
\begin{verbatim}
int pushit(int type, FILE *sfile, char *fmt, int *fscanret);
\end{verbatim}
Pushit take a type, a stream, and an one-item format string.
pushit checks the S-Lang type to conversion-type relation. It then
uses fscanf to do the conversion and pushes the items on the
stack depending on conversion flags.
fscanret holds fscanf's return value.


from: {\tt scanf.c}

from file: {\tt doc}


\paragraph{\index{scan\_gp\_cmd}scan\_gp\_cmd}
\begin{verbatim}
char *scan_gp_cmd(char *cmd)
\end{verbatim}
Scans the string cmd for special chars. Everything
between two \verb1\1@ chars is treated as an S-Lang matrix variable
to be plotted. Names delimited by the \$'s are replaced by
the string representation of that S-Lang variable.
The variable is dumped to a file named {\tt tmp\_prefix+variable\_name.\_slmat}.


Bugs: temporary files are not cleaned.


from: {\tt plot.c}

from file: {\tt doc}


\paragraph{\index{setsig}setsig}
\begin{verbatim}
void setsig()
\end{verbatim}
set up custom signal handling. Works on UNIX systems, rather broken
on other systems 


from: {\tt sigs.c}

from file: {\tt doc}


\paragraph{\index{showusage}showusage}
\begin{verbatim}
void showusage( char *argv, int verb)
\end{verbatim}
shows vamps program indentification and a short (verb $=$$=$ 0 )
or long (verb $>$0 ) command line explanation.


from: {\tt vamps.c}

from file: {\tt doc}


\paragraph{\index{sl\_fscanf}sl\_fscanf}
\begin{verbatim}
int sl_fscanf(FILE *sfile, char *fmt, char *s)
\end{verbatim}
Breaks the fmt string into single-item strings for pushit or
spushit to process. If FILE $=$$=$ NULL spushit is called, otherwise
pushit is called.


return: the number of items pushed on the stack.


from: {\tt scanf.c}

from file: {\tt doc}


\paragraph{\index{sl\_make\_strv}sl\_make\_strv}
char **sl\_make\_strv(char *str, char *sep, int *vlen)
Returns malloc'ed char ** of vlen malloc'ed char * from
string str, separated by one or more characters from sep;
nf is minimum 1, str is not modified, trailing newline is
stripped. Use sl\_free\_strv() to free memory.


from: {\tt misc.c}

from file: {\tt doc}


\paragraph{\index{spushit}spushit}
\begin{verbatim}
int spushit(int type, char *s, char *fmt, int *sscanret, int *startpos);
\end{verbatim}
spushit take a type, a string, an one-item format string and a start
position (needed because we uses multiple calls to sscanf);
pushit checks the S-Lang type to conversion-type relation. It then
uses sscanf to do the conversion and pushes the items on the
stack depending on conversion flags.
sscanret holds sscanf's return value.


from: {\tt scanf.c}

from file: {\tt doc}


\paragraph{\index{unsetsig}unsetsig}
\begin{verbatim}
int unsetsig()
\end{verbatim}
resets signal handling


from: {\tt sigs.c}

from file: {\tt doc}
nd{document}
